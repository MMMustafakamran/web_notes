\documentclass[8pt, a4paper, landscape]{extarticle}

\usepackage[left=0.1cm,right=0.1cm,top=0.1cm,bottom=0.1cm]{geometry}
\usepackage{multicol}
\usepackage{xcolor}
\usepackage{listings}
\usepackage{enumitem}
\usepackage{titlesec}
\usepackage{amsmath}

% --- Visual Setup ---
\setlength{\parindent}{0pt}
\setlength{\parskip}{0pt}
\setlength{\columnsep}{0.2cm}

% --- Code Formatting (Compact) ---
\definecolor{codegreen}{rgb}{0,0.5,0}
\definecolor{codeblue}{rgb}{0.0,0.0,0.8}
\definecolor{codepurple}{rgb}{0.58,0,0.82}

\lstset{
    language=JavaScript,
    backgroundcolor=\color{white},
    basicstyle=\ttfamily\scriptsize,
    commentstyle=\color{codegreen}\scriptsize,
    keywordstyle=\color{codeblue}\scriptsize,
    stringstyle=\color{codepurple}\scriptsize,
    breaklines=true,
    tabsize=2,
    frame=single,
    framerule=0.4pt,
    xleftmargin=1pt,
    xrightmargin=1pt,
    aboveskip=1pt,
    belowskip=1pt,
    showstringspaces=false,
    escapeinside={(*@}{@*)}
}

% --- Unified Heading Numbering with Compact Grey Box ---
\newcounter{unifiedsection}
\newcommand{\compactsection}[1]{%
    \stepcounter{unifiedsection}% increment counter
    \vspace{1pt}% top spacing
    \noindent%
    {\setlength{\fboxsep}{1pt}% reduce padding inside colorbox
    \colorbox{black!10}{%
        \makebox[\dimexpr\linewidth-2\fboxsep\relax][l]{\textbf{\footnotesize \theunifiedsection. #1}}%
    }}%
    \vspace{1pt}% bottom spacing
}

% --- Major Topic Header ---
\newcommand{\topicsection}[1]{%
    \vspace{4pt}% top spacing
    \noindent%
    {\setlength{\fboxsep}{2pt}%
    \colorbox{black}{%
        \makebox[\dimexpr\linewidth-2\fboxsep\relax][c]{\color{white}\textbf{\small #1}}%
    }}%
    \vspace{2pt}% bottom spacing
}

\begin{document}

\begin{multicols*}{3}

\topicsection{EXPRESS JS}

\compactsection{Core Server Setup}
To start a server, you need to require Express, initialize it, and listen on a port.
\begin{lstlisting}
const express = require('express');
const app = express();
const PORT = 3000;

// Basic Route
app.get('/', (req, res) => res.send('Hello World!'));

app.listen(PORT, () => console.log(`Server running on port ${PORT}`));
\end{lstlisting}

\compactsection{Route Path Patterns (Exam Heavy)}
Express uses string patterns and Regex for route matching.\\
\textbf{ab?cd} $\to$ ? means b is optional e.g $\to$ /acd, /abcd\\
\textbf{ab+cd} $\to$ + means b repeats 1 or more e.g /abcd, /abbcd\\
\textbf{ab*cd} $\to$ * means anything (0 or more) between e.g /abxcd, /ab123cd\\
\textbf{/.*fly\$/} $\to$ .* = anything, \$ = ends with "fly" $\to$ /butterfly, /dragonfly\\
\textbf{:id} $\to$ : marks route parameter $\to$ /user/34 (req.params.id = 34)

\compactsection{Middleware Execution Flow (Exam Focus)}
Middleware func have to req, res, next. They execute in the order they are defined. If \texttt{next()} not called /res not sent to client, the request hangs and routes are never reached.\\
\textbf{The "t1/t2" Analysis (From Question 4):}\\
If a request is made to /about: (order= globalmiddleware with next, routespecific funcs up to down)
\begin{enumerate}[leftmargin=*,noitemsep,topsep=0pt]
    \item \texttt{app.use((req, res, next) \to \{ console.log("t1"); next(); \});} $\to$ output: t1, continues.
    \item \texttt{app.get("/about1", ...)} $\to$ Path doesn't match, skipped.
    \item \texttt{app.use((req, res, next) \to \{ console.log("t2"); \});} $\to$ Logs t2. No \texttt{next()} is called!
    \item Execution Stops. The specific \texttt{app.get("/about")} lower in the code is never reached.
\end{enumerate}

\compactsection{Types of Middleware}
\begin{enumerate}[leftmargin=*,noitemsep,topsep=0pt]
    \item \textbf{Application-level:} Bound to \texttt{app.use()}. Runs for all requests.
    \item \textbf{Router-level:} Bound to \texttt{express.Router()}. Used to modularize code (e.g., all \texttt{/api} routes).
    \item \textbf{Built-in:} \texttt{express.static()} (files), \texttt{express.json()} (parsing JSON body).
    \item \textbf{Error-handling:} Takes four arguments: \texttt{(err, req, res, next)}.
\end{enumerate}

\compactsection{Response Methods}
Use these to end the request-response cycle.
\begin{itemize}[leftmargin=*,noitemsep,topsep=0pt]
    \item \textbf{res.send('Hello World')} – Sends basic text or HTML.
    \item \textbf{res.json(\{ message: 'Success' \})} – Sends a JSON object (best for APIs).
    \item \textbf{res.render('index', \{ title: 'Home Page' \})} – Renders an EJS or template file with parameters.
    \item \textbf{res.status(404).send('Not Found')} – Sets the HTTP status code and optionally sends a message.
    \item \textbf{res.redirect('/login')} – Redirects the user to a different URL.
\end{itemize}

\compactsection{EJS Templating}
Node.js engine to write html pages dynamically. Server keeps data (e.g., books) and sends it to the EJS page which mixes it with HTML and sends it to the browser. Set engine and use \texttt{res.render(filename, data)}.\\
\textbf{The Controller (app.js):} (Part that decides what data to send)
\begin{lstlisting}
app.set('view engine', 'ejs');
app.get('/showBooks', (req, res) => {
    const books = [{ title: "Node JS", price: 20 }];
    res.render('books', { books: books }); 
});
\end{lstlisting}
\textbf{The View (views/books.ejs):} (HTML page that shows data)
\begin{lstlisting}[language=HTML]
<ul>
  <% books.forEach(function(book) { %> 
    <li><%= book.title %> - $<%= book.price %></li> 
  <% }); %>
</ul>
\end{lstlisting}

\compactsection{Handling Data (JSON vs. MongoDB)}
Exam asks for difference between local arrays and database calls.\\
\textbf{Local JSON Approach:}
\begin{lstlisting}
app.get('/getUsers', (req, res) => {
    res.json(usersArray); // Immediate response
});
\end{lstlisting}
\textbf{MongoDB (Mongoose) Approach:}
\begin{lstlisting}
app.get('/getUsers', async (req, res) => {
    try {
        const users = await User.find({}); // Must be awaited
        res.json(users);
    } catch (err) {
        res.status(500).send(err.message);
    }
});
\end{lstlisting}

\topicsection{MONGODB \& MONGOOSE}

\compactsection{MongoDB: Core Concepts}
NoSQL, open-source, document-oriented database written in C++.\\
\textbf{Database:} Physical container for collections.\\
\textbf{Collection:} Group of documents; schema not enforced.\\
\textbf{Document:} Set of key-value pairs with dynamic schema.\\
\textbf{CRUD:} Create, Read, Update, Delete.

\compactsection{RDBMS vs. MongoDB Terminology}
\begin{itemize}[leftmargin=*,noitemsep,topsep=0pt]
    \item Table $\to$ Collection
    \item Tuple / Row $\to$ Document
    \item Column $\to$ Field
    \item Table Join $\to$ Embedded Docs / \texttt{\$lookup}
    \item Primary Key $\to$ Primary Key (default \texttt{\_id})
\end{itemize}

\compactsection{Basic Shell Commands}
\textbf{Switch/Create DB:} \texttt{use DATABASE\_NAME}\\
\textbf{Drop DB:} \texttt{db.dropDatabase()}\\
\textbf{Create Collection:} \texttt{db.createCollection("name")}\\
\textbf{Remove Collection:} \texttt{db.collection.drop()}\\
\textbf{Retrieve All Docs:} \texttt{db.collection.find()}\\
\textbf{\textit{Exam Tip:}} \texttt{db.collection.copy()} is \textbf{NOT} valid.

\compactsection{Query Operators}
Used within \texttt{find()}, \texttt{update()}, or \texttt{remove()}.\\
\textbf{\$gt:} Greater Than. E.g. \texttt{db.users.find(\{ age: \{ \$gt: 25 \} \})}\\
\textbf{\$set:} Modify specific fields without replacing whole doc.\\
\textbf{Regex:} Query strings. E.g. \texttt{\{ address: /\^S/ \}} (starts with "S").

\compactsection{Mongoose Integration (Exam Focus)}
\textbf{Connection \& Schema Setup:}
\begin{lstlisting}
const mongoose = require('mongoose');
mongoose.connect('mongodb://localhost:27017/examDB')
  .then(() => console.log('Connected!'));

const taskSchema = new mongoose.Schema({
  title: String,
  completed: Boolean
});
const Task = mongoose.model('Task', taskSchema);
\end{lstlisting}
\textbf{Exam Scenario: GET /tasks (completed is false)}
\begin{lstlisting}
app.get('/tasks', async (req, res) => {
  try {
    const pendingTasks = await Task.find({ completed: false }); 
    res.json(pendingTasks);
  } catch (err) {
    res.status(500).json({ error: err.message });
  }
});
\end{lstlisting}

\compactsection{Relationships in Mongoose}
\textbf{One-to-One (Nesting):}
\begin{lstlisting}
const userSchema = new mongoose.Schema({
  name: String,
  passport: { number: String, expiryDate: Date }
});
\end{lstlisting}
\textbf{One-to-Many (References):}
\begin{lstlisting}
const postSchema = new mongoose.Schema({
  content: String,
  author: { type: mongoose.Schema.Types.ObjectId, ref: 'User' }
});
// Join:
const posts = await Post.find().populate('author');
\end{lstlisting}
\textbf{Many-to-Many (Junction):}
\begin{lstlisting}
const enrollmentSchema = new mongoose.Schema({
  student: { type: mongoose.Schema.Types.ObjectId, ref: 'Student' },
  course: { type: mongoose.Schema.Types.ObjectId, ref: 'Course' }
});
\end{lstlisting}

\compactsection{Advanced Node.js Operations}
Native mongodb driver (if asked):\\
\textbf{Insert One:} \texttt{dbo.collection("customers").insertOne(obj, cb)}\\
\textbf{Sort:} \texttt{find().sort(\{ name: 1 \})} (1 Asc, -1 Desc)\\
\textbf{Limit:} \texttt{find().limit(5)}\\
\textbf{Join (\$lookup):} Used in aggregation for left outer joins.

\topicsection{NODE.JS FUNDAMENTALS}

\compactsection{Node.js Fundamentals}
Asynchronous, single-threaded, event-driven runtime (V8 engine).\\
\textbf{Traditional (Apache/IIS):} Multi-threaded.\\
\textbf{Libuv:} C++ library managing thread pool for I/O tasks.\\
\textbf{NPM \& Project Setup:}\\
\texttt{npm init} (initialize), \texttt{npm install -g} (global).\\
\textbf{dependencies:} Production (\texttt{--save}).\\
\textbf{devDependencies:} Testing/Dev (\texttt{--save-dev}).\\
\textbf{package.json:} Metadata. \textbf{package-lock.json:} Exact versions.\\
\textbf{npm ci:} Exact versions from lockfile (CI/CD).

\compactsection{Global \& Utility Modules}
\textbf{OS:} \texttt{os.platform()}, \texttt{os.totalmem()}, \texttt{os.type()}\\
\textbf{Path:} \texttt{path.resolve()}, \texttt{path.extname()}\\
\textbf{DNS:} \texttt{dns.lookup('google.com', callback)}\\
\textbf{Console:} \texttt{console.time('label')} / \texttt{console.timeEnd('label')}\\
\textbf{Globals:} \texttt{\_\_dirname} (folder), \texttt{\_\_filename} (file)
\begin{lstlisting}
const os = require('os');
const path = require('path');
const dns = require('dns');
console.time('run');
console.log('Platform:', os.platform());
console.log('Extension:', path.extname(__filename));
dns.lookup('google.com', (err, ip) => { console.log('IP:', ip); });
console.log('Dir:', __dirname);
console.timeEnd('run');
\end{lstlisting}

\compactsection{Debugging \& Callbacks}
\textbf{Console:} \texttt{console.trace()} (call stack).\\
\textbf{Callbacks:} "Error-First" convention \texttt{function(err, data) \{ ... \}}.
\begin{lstlisting}
console.time('calc');
let sum = 0;
for (let i = 0; i < 1000000; i++) sum += i;
console.timeEnd('calc');

function a() { b(); }
function b() { console.trace(); }
a();

function fetchData(callback) {
    setTimeout(() => {
        const error = Math.random() > 0.5 ? null : new Error('Failed');
        callback(error, error ? null : 'Data');
    }, 100);
}
fetchData((err, data) => {
    if (err) console.error(err.message);
    else console.log(data);
});
\end{lstlisting}

\end{multicols*}

\end{document}
