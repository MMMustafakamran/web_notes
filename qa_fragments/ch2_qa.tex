\subsection*{Comprehensive Exam Prep: CSS}

\begin{tcolorbox}[colback=red!5,colframe=red!75,title=Past Paper Questions (Sessional \& Final)]
\begin{enumerate}
    \item \textbf{MCQ:} Which CSS property is used to change text color?
    \begin{enumerate}[label=\alph*.]
        \item font-style
        \item color
        \item text-color
        \item background
    \end{enumerate}
    \textbf{Answer:} b) color

    \item \textbf{MCQ:} In CSS, which of the following selectors has the highest specificity?
    \begin{enumerate}[label=\alph*.]
        \item .card
        \item div p
        \item \#header
        \item p
    \end{enumerate}
    \textbf{Answer:} c) \#header (IDs have higher specificity than classes or tags).

    \item \textbf{Snippet:} Write CSS to style \texttt{.btn-primary} with navy background, white text, 10px 20px padding, and no border.
\begin{lstlisting}[language=CSS]
.btn-primary {
    background-color: navy;
    color: white;
    padding: 10px 20px;
    border: none;
}
\end{lstlisting}
\end{enumerate}
\end{tcolorbox}

\begin{tcolorbox}[colback=gray!5,colframe=gray!75,title=Extra Practice Questions]
\subsubsection*{Multiple Choice Questions}
\begin{enumerate}[resume]
    \item How do you make a list that lists its items with squares?
    \begin{enumerate}[label=\alph*.]
        \item \texttt{list: square;}
        \item \texttt{list-type: square;}
        \item \texttt{list-style-type: square;}
        \item \texttt{list-style-image: square;}
    \end{enumerate}
    \textbf{Answer:} c) \texttt{list-style-type: square;}

    \item What is the default value of the \texttt{position} property?
    \begin{enumerate}[label=\alph*.]
        \item relative
        \item fixed
        \item absolute
        \item static
    \end{enumerate}
    \textbf{Answer:} d) static
\end{enumerate}

\subsubsection*{Advanced Styling Concepts}
\textbf{Question:} Explain the CSS Box Model with a diagrammatic description.
\newline \textbf{Answer:} The Box Model consists of four layers surrounding the HTML element:
\begin{itemize}
    \item \textbf{Content}: The actual text or image.
    \item \textbf{Padding}: Transparent area between content and border.
    \item \textbf{Border}: A line going around the padding and content.
    \item \textbf{Margin}: Transparent area outside the border, separating the element from others.
\end{itemize}

\textbf{Question:} Center a \texttt{div} horizontally and vertically using Flexbox.
\begin{lstlisting}[language=CSS]
.container {
    display: flex;
    justify-content: center; /* Horizontal */
    align-items: center;     /* Vertical */
    height: 100vh;
}
\end{lstlisting}
\end{tcolorbox}
