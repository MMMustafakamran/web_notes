\subsection*{Comprehensive Exam Prep: Express.js}

\begin{tcolorbox}[colback=magenta!5,colframe=magenta!75,title=Past Paper Questions]
\begin{enumerate}
    \item \textbf{MCQ:} Which method is used to define a GET route in Express?
    \textbf{Answer:} \texttt{app.get()}

    \item \textbf{MCQ:} Which middleware function is used to parse incoming request bodies?
    \textbf{Answer:} \texttt{body-parser} (or built-in \texttt{express.json()} in modern versions).

    \item \textbf{Route Logic:} What is the output of the following sequence?
    \begin{lstlisting}[language=JavaScript]
    app.use((req, res, next) => { console.log('A'); next(); });
    app.get('/test', (req, res) => { console.log('B'); });
    \end{lstlisting}
    \textbf{Answer:} Output will be \texttt{A} then \texttt{B}. The middleware runs for all routes, then \texttt{next()} passes control to the specific GET handler.
\end{enumerate}
\end{tcolorbox}

\begin{tcolorbox}[colback=gray!5,colframe=gray!75,title=Extra Practice Questions]
\subsubsection*{Middleware \& Routing}
\textbf{Question:} Write a custom middleware that logs the Request URL and Method.
\begin{lstlisting}[language=JavaScript]
app.use((req, res, next) => {
    console.log(`Method: ${req.method}, URL: ${req.url}`);
    next(); // Critical to prevent hanging
});
\end{lstlisting}

\textbf{Question:} How do you access the query parameter \texttt{?id=123} in Express?
\newline \textbf{Answer:} \texttt{req.query.id}

\textbf{Question:} How do you access the URL parameter \texttt{/users/:uId} (where uId is 123)?
\newline \textbf{Answer:} \texttt{req.params.uId}
\end{tcolorbox}
