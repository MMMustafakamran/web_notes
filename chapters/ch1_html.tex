\section{Chapter 1: HTML - HyperText Markup Language}

This document provides comprehensive teaching notes for HTML, covering everything from basic structure to advanced semantic elements and media integration.


\begin{tcolorbox}[colback=red!5,colframe=red!75,title=Exam Tips \& Hints]
\begin{itemize}
    \item \textbf{Semantic Tags}: Always use semantic tags like \texttt{<header>}, \texttt{<nav>}, and \texttt{<footer>} instead of just \texttt{<div>}. This is a frequent exam question.
    \item \textbf{Tables}: Practice \texttt{rowspan} and \texttt{colspan} thoroughly. Designers often test your ability to structure complex tables from a visual diagram.
    \item \textbf{Attributes}: Remember that \texttt{alt} is mandatory for accessibility in \texttt{<img>} tags.
\end{itemize}
\end{tcolorbox}

\subsection{Introduction to HTML}
HTML stands for \textbf{HyperText Markup Language}. It was created by Tim Berners-Lee in 1989 to distribute information across a network of computers.

\begin{itemize}
    \item \textbf{Hypertext}: Documents contain links that allow jumping to other places or documents.
    \item \textbf{Markup}: Uses tags and attributes to define the structure and presentation of data.
    \item \textbf{Interpretation}: Directly interpreted by the web browser.
    \item \textbf{Case Sensitivity}: HTML is \textbf{not} case-sensitive. Multiple spaces are ignored.
\end{itemize}

\subsection{Basic Construction of an HTML Page}
Every HTML5 document should follow this basic structure:

\end{lstlisting}

\begin{tcolorbox}[colback=blue!5,colframe=blue!75,title=Practical Example: Basic HTML Page]
A simple page demonstrating headings, lists, and links:
\begin{lstlisting}[language=HTML]
<!DOCTYPE html>
<html lang="en">
<head>
    <meta charset="UTF-8">
    <title>My First Web Page</title>
</head>
<body>
    <h1>Welcome to My Portfolo</h1>
    <p>This is a small demonstration of <strong>HTML fundamentals</strong>.</p>
    
    <h2>My Skills</h2>
    <ul>
        <li>HTML5 (Structure)</li>
        <li>CSS3 (Design)</li>
        <li>JavaScript (Interactivity)</li>
    </ul>

    <h2>Projects</h2>
    <ol>
        <li><a href="https://github.com/user/project1">Weather App</a></li>
        <li><a href="project2.html">Personal Blog</a></li>
    </ol>
</body>
</html>
\end{lstlisting}
\end{tcolorbox}

\subsubsection{Key tags in the structure:}
\begin{itemize}
    \item \texttt{<!DOCTYPE html>}: Tells the browser to expect HTML5.
    \item \texttt{<html>}: The root element that wraps all content.
    \item \texttt{<head>}: Contains metadata (title, character set, links to external files) not visible on the page.
    \item \texttt{<body>}: Contains the visible content of the web page.
\end{itemize}

\subsection{Elements, Tags, and Attributes}
\begin{itemize}
    \item \textbf{Element}: Defined by a start tag, content, and an end tag (e.g., \texttt{<p>Text</p>}).
    \item \textbf{Tag}: The angle brackets surrounding an element name. Most occur in pairs (opening and closing).
    \item \textbf{Attribute}: Provides additional information about an element (e.g., \texttt{id}, \texttt{class}, \texttt{src}, \texttt{href}). Defined within the opening tag.
\end{itemize}

\subsection{Text Formatting and Headings}
\subsubsection{Headings}
HTML provides six levels of headings from \texttt{<h1>} (most important) to \texttt{<h6>} (least important). Search engines use these to understand page hierarchy.

\subsubsection{Text Elements}
\begin{itemize}
    \item \texttt{<p>}: Paragraph.
    \item \texttt{<b>} or \texttt{<strong>}: Bold/Strong importance.
    \item \texttt{<i>} or \texttt{<em>}: Italic/Emphasized text.
    \item \texttt{<mark>}: Highlighted text.
    \item \texttt{<small>}: Smaller text.
    \item \texttt{<strike>}: Strikethrough text.
    \item \texttt{<u>} or \texttt{<ins>}: Underlined/Inserted text.
    \item \texttt{<sub>} and \texttt{<sup>}: Subscript and Superscript.
\end{itemize}

\subsubsection{Special Characters}
Use entities for symbols not on the keyboard:
\begin{itemize}
    \item \texttt{\&nbsp;}: Non-breaking space
    \item \texttt{\&copy;}: \copyright
    \item \texttt{\&reg;}: \textregistered
    \item \texttt{\&euro;}: \texteuro
    \item \texttt{\&gt;}: $>$
\end{itemize}

\subsection{Lists}
\subsubsection{Unordered Lists (\texttt{<ul>})}
Used for items where order doesn't matter.
\begin{itemize}
    \item Each item is wrapped in \texttt{<li>}.
    \item Attributes: \texttt{type} (circle, square, disc).
\end{itemize}

\subsubsection{Ordered Lists (\texttt{<ol>})}
Used for sequential data.
\begin{itemize}
    \item Attributes: \texttt{type} (1, A, a, I, i), \texttt{start}, \texttt{reversed}.
\end{itemize}

\subsection{Links and Navigation}
\subsubsection{Types of Links}
\begin{itemize}
    \item \textbf{Internal Links}: Link to another page in the same website using a \textbf{relative path}.
    \item \textbf{External Links}: Link to a different website using an \textbf{absolute path} (starting with \texttt{http://} or \texttt{https://}).
\end{itemize}

\subsubsection{Implementation}
\begin{lstlisting}[language=HTML]
<a href="destination.html" target="_blank">Clickable Text</a>
\end{lstlisting}
\begin{itemize}
    \item \texttt{href}: The destination URL.
    \item \texttt{target="\_blank"}: Opens the link in a new tab.
\end{itemize}

\subsection{Media: Images, Video, and Audio}
    \item \textbf{Image Maps}: Allow multiple links (hotspots) on a single image using \texttt{<map>} and \texttt{<area>}.
\end{itemize}

\begin{tcolorbox}[colback=orange!5,colframe=orange!75,title=Practical Example: Image Maps]
Defining clickable regions on a local image:
\begin{lstlisting}[language=HTML]
<img src="world-map.jpg" alt="World Map" usemap="#mapname">

<map name="mapname">
  <!-- Circular hotspot -->
  <area shape="circle" coords="100,100,50" href="europe.html" alt="Europe">
  <!-- Rectangular hotspot -->
  <area shape="rect" coords="200,50,400,150" href="asia.html" alt="Asia">
</map>
\end{lstlisting}
\end{tcolorbox}

\subsubsection{Video and Audio (HTML5)}
\begin{lstlisting}[language=HTML]
<video controls autoplay loop muted>
  <source src="movie.mp4" type="video/mp4">
</video>

<audio controls>
  <source src="song.mp3" type="audio/mpeg">
</audio>
\end{lstlisting}
Attributes: \texttt{controls}, \texttt{autoplay}, \texttt{muted}, \texttt{loop}.

\subsection{HTML Tables}
Used to display data in a grid (rows and columns).

\begin{itemize}
    \item \texttt{<table>}: Starts the table.
    \item \texttt{<tr>}: Table Row.
    \item \texttt{<td>}: Table Data (cell).
    \item \texttt{<th>}: Table Header cell (bold and centered by default).
    \texttt{<caption>}: Adds a title to the table.
\end{itemize}

    \item \texttt{rowspan} and \texttt{colspan}: Allow cells to span multiple rows or columns.
\end{itemize}

\begin{tcolorbox}[colback=green!5,colframe=green!75,title=Practical Example: Advanced Table Layout]
A complex schedule table using spanning:
\begin{lstlisting}[language=HTML]
<table border="1">
  <tr>
    <th rowspan="2">Day</th>
    <th colspan="2">Activities</th>
  </tr>
  <tr>
    <th>Morning</th>
    <th>Evening</th>
  </tr>
  <tr>
    <td>Monday</td>
    <td>Coding</td>
    <td>Gym</td>
  </tr>
  <tr>
    <td>Tuesday</td>
    <td colspan="2">Project Deadline (Full Day)</td>
  </tr>
</table>
\end{lstlisting}
\end{tcolorbox}

\subsection{HTML Forms}
Used to collect user input and send it to a server.

\begin{itemize}
    \item \texttt{<form>}: Wraps the form elements.
    \begin{itemize}
        \item \texttt{action}: URL where data is sent.
        \item \texttt{method}: \texttt{GET} (visible in URL) or \texttt{POST} (secure/hidden).
    \end{itemize}
    \item \textbf{Common Elements}:
    \begin{itemize}
        \item \texttt{<input>}: versatile input field (type="text", "password", "submit", etc.).
        \item \textbf{HTML5 Enhancements}: \texttt{type="email"}, \texttt{type="date"}, \texttt{type="number"}, \texttt{type="color"}. These provide built-in validation.
    \end{itemize}
\end{itemize}

\subsection{Semantic HTML and Accessibility}
Semantic HTML gives meaning to code, helping search engines and screen readers.

\subsubsection{Block vs. Inline}
\begin{itemize}
    \item \textbf{Block-level}: Starts on a new line, takes full width (e.g., \texttt{<div>}, \texttt{<h1>}, \texttt{<p>}, \texttt{<ul>}).
    \item \textbf{Inline-level}: Stays in the flow, only takes necessary width (e.g., \texttt{<span>}, \texttt{<a>}, \texttt{<img>}).
\end{itemize}

\end{itemize}

\subsection*{Comprehensive Exam Prep: HTML}

\begin{tcolorbox}[colback=blue!5,colframe=blue!75,title=Past Paper Questions (Sessional \& Final)]
\begin{enumerate}
    \item \textbf{MCQ:} Which HTML attribute is used to define inline styles? \textit{(Sessional 1 2025)}
    \begin{enumerate}[label=\alph*.]
        \item style
        \item class
        \item styles
        \item font
    \end{enumerate}
    \textbf{Answer:} a) style

    \item \textbf{MCQ:} What is the correct HTML element for the largest heading? \textit{(Final 2024)}
    \begin{enumerate}[label=\alph*.]
        \item \texttt{<h1>}
        \item \texttt{<h6>}
        \item \texttt{<head>}
        \item \texttt{<header>}
    \end{enumerate}
    \textbf{Answer:} a) \texttt{<h1>}

    \item \textbf{MCQ:} Which HTML attribute is used to provide a unique identifier for an element?
    \textbf{Answer:} a) id

    \item \textbf{MCQ:} What does the \texttt{<ol>} tag represent in HTML?
    \textbf{Answer:} a) Ordered List

    \item \textbf{Fill in the blank:} To make a hyperlink open in a new browser tab, the attribute of the \texttt{<a>} tag is \textbf{target} and should be set to \textbf{\_blank}.
\end{enumerate}
\end{tcolorbox}

\begin{tcolorbox}[colback=gray!5,colframe=gray!75,title=Extra Practice Questions]
\subsubsection*{Multiple Choice Questions}
\begin{enumerate}[resume]
    \item Which tag is used to create a drop-down list?
    \begin{enumerate}[label=\alph*.]
        \item \texttt{<input type="dropdown">}
        \item \texttt{<list>}
        \item \texttt{<select>}
        \item \texttt{<option>}
    \end{enumerate}
    \textbf{Answer:} c) \texttt{<select>}

    \item Which semantic tag determines the footer of a document or section?
    \begin{enumerate}[label=\alph*.]
        \item \texttt{<bottom>}
        \item \texttt{<footer>}
        \item \texttt{<section>}
        \item \texttt{<aside>}
    \end{enumerate}
    \textbf{Answer:} b) \texttt{<footer>}
\end{enumerate}

\subsubsection*{Code Analysis}
\textbf{Question:} Write the HTML code to create a form that sends data to \texttt{/submit} using the \texttt{POST} method. It should contain a text field for "Name" and a submit button.
\begin{lstlisting}[language=HTML]
<form action="/submit" method="POST">
    <label for="name">Name:</label>
    <input type="text" id="name" name="user_name">
    <button type="submit">Submit</button>
</form>
\end{lstlisting}

\subsubsection*{Theory Short Questions}
\begin{itemize}
    \item \textbf{Difference between Block and Inline elements:}
    \begin{itemize}
        \item \textbf{Block}: Starts on a new line and takes up full width (e.g., \texttt{div}, \texttt{p}, \texttt{h1}).
        \item \textbf{Inline}: Starts on the same line and takes only necessary width (e.g., \texttt{span}, \texttt{a}, \texttt{img}).
    \end{itemize}
    \item \textbf{Purpose of the \texttt{alt} attribute:} 
    \newline Provides alternative text for screen readers (accessibility) and displays if the image fails to load.
\end{itemize}
\end{tcolorbox}




