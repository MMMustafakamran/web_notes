\section{Chapter 8: Node.js - Server-Side JavaScript Runtime}

Node.js is a powerful JavaScript platform built on Chrome's V8 engine. It allows developers to use JavaScript to write server-side code, enabling "Full-stack JavaScript" development.

\begin{tcolorbox}[colback=red!5,colframe=red!75,title=Exam Tips \& Hints]
\begin{itemize}
    \item \textbf{Event Loop}: A fundamental concept. Understand the roles of the Call Stack, libuv, and the Task Queue.
    \item \textbf{NPM}: Know the difference between \texttt{dependencies} and \texttt{devDependencies}.
    \item \textbf{Global Objects}: Remember \texttt{\_\_dirname} and \texttt{\_\_filename} are provided by the Node.js wrapper, not part of standard JS.
    \item \textbf{Callback Pattern}: Error-first callbacks (\texttt{err} as the first argument) is a standard convention.
\end{itemize}
\end{tcolorbox}

\subsection{Core Principles}
\begin{itemize}
    \item \textbf{Asynchronous \& Event-Driven}: All APIs of Node.js library are asynchronous (non-blocking). A Node-based server never waits for an API to return data.
    \item \textbf{Single-Threaded}: Node.js uses a single-threaded model with event looping. This makes it highly scalable for I/O intensive apps, though it's not ideal for CPU-intensive tasks.
    \item \textbf{Platform Independent}: Runs on Windows, Linux, and macOS.
\end{itemize}

\subsection{The Node.js Event Loop}
\begin{itemize}
    \item \textbf{libuv}: A library that Node.js uses to perform asynchronous I/O and manage a thread pool.
    \item \textbf{Call Stack}: Where the code being executed lives.
    \item \textbf{Event Queue}: When an async task (like a timer or file read) completes, its callback is sent to this queue.
    \item \textbf{Event Loop}: Monitors the call stack and event queue. If the stack is empty, it pushes the first task from the queue to the stack for execution.
\end{itemize}

\subsection{Node Package Manager (NPM)}
NPM is the default package manager for Node.js.
\begin{itemize}
    \item \textbf{\texttt{package.json}}: Holds metadata about the project and its dependencies.
    \item \textbf{\texttt{package-lock.json}}: Records the exact version of every installed dependency to ensure consistent builds across environments.
    \item \textbf{Commands}:
    \begin{itemize}
        \item \texttt{npm init}: Initialize a new project.
        \item \texttt{npm install <pkg>}: Install a package and add it to \texttt{dependencies}.
        \item \texttt{npm ci}: A cleaner, faster install for automated environments (CI/CD).
        \item \texttt{npm audit}: Checks for security vulnerabilities in dependencies.
    \end{itemize}
\end{itemize}

\subsection{Global Objects and Modules}
\begin{itemize}
    \item \textbf{\texttt{\_\_dirname}}: Directory name of the current script.
    \item \textbf{\texttt{\_\_filename}}: File name of the current script.
    \item \textbf{\texttt{process}}: Provides information about, and control over, the current Node.js process.
    \item \textbf{\texttt{console}}: Used for printing to stdout and stderr.
\end{itemize}

    \item \textbf{Net}: Used to create TCP servers and clients.
\end{enumerate}

\begin{tcolorbox}[colback=cyan!5,colframe=cyan!75,title=Practical Example: Reading and Writing Files]
Using the built-in \texttt{fs} module (promises version):
\begin{lstlisting}[language=JavaScript]
const fs = require('fs').promises;

async function manageFiles() {
  try {
    // Write to a file
    await fs.writeFile('test.txt', 'Hello Node.js!');
    // Read from the file
    const data = await fs.readFile('test.txt', 'utf8');
    console.log(data);
  } catch (err) {
    console.error(err);
  }
}
manageFiles();
\end{lstlisting}
\end{tcolorbox}

\begin{tcolorbox}[colback=purple!5,colframe=purple!75,title=Practical Example: Simple HTTP Server]
Creating a server without any frameworks:
\begin{lstlisting}[language=JavaScript]
const http = require('http');

const server = http.createServer((req, res) => {
  res.statusCode = 200;
  res.setHeader('Content-Type', 'text/plain');
  res.end('Hello from Node.js Server!');
});

server.listen(3000, '127.0.0.1', () => {
  console.log('Server running at http://127.0.0.1:3000/');
});
\end{lstlisting}
\end{tcolorbox}

\subsection*{Comprehensive Exam Prep: Node.js (Advanced)}

\begin{tcolorbox}[colback=brown!5,colframe=brown!75,title=Past Paper Questions]
\begin{enumerate}
    \item \textbf{MCQ:} What is the default port for a Node.js application (conventionally)?
    \textbf{Answer:} 3000 (though it can be anything distinct).

    \item \textbf{MCQ:} Which command is used to initialize a new Node.js project?
    \textbf{Answer:} \texttt{npm init}

    \item \textbf{MCQ:} Which of the following is not a valid JS data type?
    \textbf{Answer:} Character (JS has String, no distinct Char type).
\end{enumerate}
\end{tcolorbox}

\begin{tcolorbox}[colback=gray!5,colframe=gray!75,title=Extra Practice Questions]
\subsubsection*{Architecture & Globals}
\textbf{Question:} Explain the Node.js Event Loop.
\newline \textbf{Answer:} Node.js is single-threaded. the Event Loop handles asynchronous callbacks. It offloads operations (like I/O) to the system kernel (libuv) and puts their callbacks into a queue. When the Call Stack is empty, the Event Loop pushes tasks from the queue to the stack.

\textbf{Question:} What is the difference between \texttt{module.exports} and \texttt{exports}?
\newline \textbf{Answer:} \texttt{exports} is a reference to \texttt{module.exports}. If you assign a new object to \texttt{exports}, you break the link. Always use \texttt{module.exports} when exporting a single class or function.

\textbf{Question:} What does \texttt{\_\_dirname} return?
\newline \textbf{Answer:} The absolute path of the directory containing the currently executing file.
\end{tcolorbox}




