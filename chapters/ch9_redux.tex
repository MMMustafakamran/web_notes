\section{Chapter 9: Redux - State Management}

Redux is a predictable state container for JavaScript apps. It helps you write applications that behave consistently, run in different environments (client, server, and native), and are easy to test.

\begin{tcolorbox}[colback=red!5,colframe=red!75,title=Exam Tips \& Hints]
\begin{itemize}
    \item \textbf{Principles}: Single source of truth (Store), State is read-only (Actions), Changes are made with pure functions (Reducers).
    \item \textbf{Data Flow}: Unidirectional data flow: Action $\rightarrow$ Reducer $\rightarrow$ Store $\rightarrow$ View.
    \item \textbf{Hooks}: Familiarize yourself with \texttt{useSelector} and \texttt{useDispatch} for modern Redux.
    \item \textbf{Immutability}: Never mutate the state directly in a reducer; always return a new object/array.
\end{itemize}
\end{tcolorbox}

\subsection{Core Concepts}
\begin{enumerate}
    \item \textbf{Store}: The centralized place where the state of the application is stored.
    \item \textbf{Actions}: Plain JavaScript objects that describe "what happened". Every action must have a \texttt{type} property.
    \item \textbf{Reducers}: Pure functions that specify how the application's state changes in response to an action.
    \item \textbf{Dispatch}: The method used to send actions to the store.
    \item \textbf{Connect}: A function (mostly used in class components) from \texttt{react-redux} to link React components to the Redux store.
\end{enumerate}

\begin{tcolorbox}[colback=blue!5,colframe=blue!75,title=Practical Example: Basic Redux Setup]
A simple counter example:
\begin{lstlisting}[language=JavaScript]
// 1. Action Types
const INCREMENT = 'INCREMENT';

// 2. Action Creator
const incrementAction = () => ({ type: INCREMENT });

// 3. Reducer
const counterReducer = (state = { count: 0 }, action) => {
  switch (action.type) {
    case INCREMENT:
      return { ...state, count: state.count + 1 };
    default:
      return state;
  }
};

// 4. Store
import { createStore } from 'redux';
const store = createStore(counterReducer);

// 5. Usage
store.dispatch(incrementAction());
console.log(store.getState()); // { count: 1 }
\end{lstlisting}
\end{tcolorbox}

\subsection*{Comprehensive Exam Prep: Redux}

\begin{tcolorbox}[colback=purple!5,colframe=purple!75,title=Past Paper Questions (Final 2024)]
\begin{enumerate}
    \item \textbf{Theory:} Explain the Redux concept using practical example code.
    \begin{itemize}
        \item \textbf{Store}: Holds the state tree.
        \item \textbf{Action}: Dispatched to trigger changes (must have \texttt{type}).
        \item \textbf{Reducer}: Pure function \texttt{(state, action) => newState}.
    \end{itemize}
    \textit{(Refer to Chapter 9 notes for the full Book List code example)}
\end{enumerate}
\end{tcolorbox}

\begin{tcolorbox}[colback=gray!5,colframe=gray!75,title=Extra Practice Questions]
\subsubsection*{Core Concepts}
\textbf{Question:} What is the "Single Source of Truth"?
\newline \textbf{Answer:} In Redux, the state of your whole application is stored in an object tree within a single store.

\textbf{Question:} Why must Reducers be "pure functions"?
\newline \textbf{Answer:} They must not mutate the existing state. They should take the previous state and an action, and return a \textit{new} state object. This makes state changes predictable and allows features like time-travel debugging.

\textbf{Task:} Write a reducer for a "Counter" that handles \texttt{INCREMENT}.
\begin{lstlisting}[language=JavaScript]
const initialState = { count: 0 };
function counterReducer(state = initialState, action) {
  switch(action.type) {
    case 'INCREMENT':
      return { ...state, count: state.count + 1 };
    default:
      return state;
  }
}
\end{lstlisting}
\end{tcolorbox}


