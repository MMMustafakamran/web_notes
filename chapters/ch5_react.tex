\section{Chapter 5: React - Modern Front-end Development}

React is a powerful JavaScript library for building user interfaces, developed by Facebook. It focuses on reusable components and efficient rendering via the Virtual DOM.

\begin{tcolorbox}[colback=red!5,colframe=red!75,title=Exam Tips \& Hints]
\begin{itemize}
    \item \textbf{Virtual DOM}: Understand that the Virtual DOM is a lightweight copy of the Real DOM. React uses it to calculate the minimum changes (Reconciliation) before updating the actual browser DOM.
    \item \textbf{Hooks}: \texttt{useState} and \texttt{useEffect} are extremely common in practical coding questions.
    \item \textbf{Class Context}: Remember that in class components, custom methods must be bound to \texttt{this} or written as arrow functions to access state. 
    \item \textbf{Data Flow}: Data flows \textit{down} via props and \textit{up} via function callbacks.
    \item \textbf{Keys}: Expect a question on why \texttt{key} props are needed when mapping lists (helping React's reconciliation process).
\end{itemize}
\end{tcolorbox}

\subsection{Core Concepts}
\subsubsection{React vs React Native}
\begin{itemize}
    \item \textbf{React}: For web applications (Single Page Apps).
    \item \textbf{React Native}: For mobile application development (Android/iOS).
\end{itemize}

\subsubsection{The Virtual DOM}
React creates an in-memory database cache (Virtual DOM) that tracks changes. When the UI changes, React compares the Virtual DOM to the real DOM and updates only the necessary parts, leading to better performance.

\subsubsection{JSX (JavaScript XML)}
A syntax extension that looks like HTML but lives in JavaScript.
\begin{itemize}
    \item Must have a single root element.
    \item Use \texttt{className} instead of \texttt{class}.
    \item Use \texttt{htmlFor} instead of \texttt{for} for labels.
\end{itemize}

    \item \textbf{Class Components (Container)}: Use ES6 classes and have access to \texttt{this}, \texttt{state}, and lifecycle methods.
\end{itemize}

\begin{tcolorbox}[colback=blue!5,colframe=blue!75,title=Practical Example: Simple Functional Component]
A reusable component that displays a greeting:
\begin{lstlisting}[language=JavaScript]
import React from 'react';

const Welcome = (props) => {
  return (
    <div className="welcome-card">
      <h1>Hello, {props.name}!</h1>
      <p>Welcome to our React application.</p>
    </div>
  );
};

export default Welcome;
\end{lstlisting}
\end{tcolorbox}

\subsection{State and Props}
\subsubsection{State}
An internal data store for a component. When state changes, the component re-renders.
\begin{itemize}
    \item Always use \texttt{this.setState()} in class components to update state.
    \item In functional components, use the \texttt{useState} hook.
\end{itemize}

\subsubsection{Props (Properties)}
Data passed from a parent component to a child component. Props are read-only for the child.

\begin{tcolorbox}[colback=green!5,colframe=green!75,title=Practical Example: State and Props]
Managing a counter with \texttt{useState}:
\begin{lstlisting}[language=JavaScript]
import React, { useState } from 'react';

function Counter() {
  const [count, setCount] = useState(0);

  return (
    <div>
      <p>You clicked {count} times</p>
      <button onClick={() => setCount(count + 1)}>
        Click me
      </button>
    </div>
  );
}
\end{lstlisting}
\end{tcolorbox}

\subsection{Handling Events}
\begin{itemize}
    \item React events are named using camelCase (e.g., \texttt{onClick}, \texttt{onSubmit}).
    \item \textbf{\texttt{this} Context}: In class components, \texttt{this} is lost in custom functions. Solutions:
    \begin{itemize}
        \item Bind in constructor: \texttt{this.myFunc = this.myFunc.bind(this)}.
        \item Use \textbf{Arrow Functions}: \texttt{myFunc = () =$>$ \{ ... \}}.
    \end{itemize}
\end{itemize}

\subsection{Forms and Interactivity}
\begin{itemize}
    \item \textbf{Controlled Components}: Input values are driven by state.
    \item \textbf{\texttt{e.preventDefault()}}: Used in form submission to stop the default page refresh.
    \item \textbf{Functions as Props}: Passing a function from parent to child allows the child to communicate back to the parent (e.g., deleting an item from a list held in parent state).
\end{itemize}

\subsection{Development Tools and Setup}
\begin{itemize}
    \item \textbf{React Dev Tools}: Browser extension for inspecting component hierarchies and state/props.
    \item \textbf{Create React App (CRA)}: A standard toolchain for setting up a modern React project.
    \begin{itemize}
        \item \texttt{npx create-react-app my-app}
        \item \texttt{npm start}: Runs the app in development mode.
    \end{itemize}
\end{itemize}

\subsection{React Router}
Used to handle navigation in a Single Page App without refreshing the page.
\begin{itemize}
    \item \textbf{\texttt{<BrowserRouter>}}: Wraps the whole app.
    \item \textbf{\texttt{<Route>}}: Defines a path and the component to render.
    \item \textbf{\texttt{<Link>}} and \texttt{<NavLink>}: Replace \texttt{<a>} tags for internal navigation.
    \item \textbf{Route Parameters}: \texttt{path="/post/:id"}.
\end{itemize}

\subsection{Advanced Topics}
\subsubsection{Higher-Order Components (HOCs)}
A function that takes a component and returns a new ("enhanced") component. Used for shared logic like authentication or loading states.
\begin{lstlisting}[language=JavaScript]
const Protected = withAuth(Dashboard);
\end{lstlisting}

\subsection*{Comprehensive Exam Prep: React}

\begin{tcolorbox}[colback=orange!5,colframe=orange!75,title=Past Paper Questions (Sessional 2 2024 \& Final)]
\begin{enumerate}
    \item \textbf{Theory:} Difference between Real DOM and Virtual DOM?
    \begin{itemize}
        \item \textbf{Real DOM}: The actual HTML structure. Slow updates (heavy reflow/repaint).
        \item \textbf{Virtual DOM}: Lightweight JavaScript object copy. React updates this first, compares (diffing), and patches only changes to Real DOM (reconciliation).
    \end{itemize}

    \item \textbf{Theory:} List common React events.
    \newline \textbf{Answer:} \texttt{onClick}, \texttt{onChange}, \texttt{onSubmit}, \texttt{onMouseEnter}.

    \item \textbf{Theory:} What are Keys? Are they necessary?
    \newline \textbf{Answer:} Keys help React identify which items in a list have changed, added, or removed. They are necessary for performance and avoiding bugs in dynamic lists.

    \item \textbf{Task:} Create a Counter Component (Class-based).
\begin{lstlisting}[language=JavaScript]
class Counter extends React.Component {
  state = { count: 0 };
  increment = () => this.setState({ count: this.state.count + 1 });
  render() {
    return <button onClick={this.increment}>{this.state.count}</button>;
  }
}
\end{lstlisting}
\end{enumerate}
\end{tcolorbox}

\begin{tcolorbox}[colback=gray!5,colframe=gray!75,title=Extra Practice Questions]
\subsubsection*{Hooks & Functional Components}
\begin{enumerate}
    \item How do you replicate \texttt{componentDidMount} with \texttt{useEffect}?
    \newline \textbf{Answer:} \texttt{useEffect(() => \{ ... \}, [])} (Empty dependency array).

    \item \textbf{Code Snippet:} Create a component using \texttt{useState} to toggle text visibility.
\begin{lstlisting}[language=JavaScript]
function Toggle() {
  const [show, setShow] = useState(true);
  return (
    <div>
      <button onClick={() => setShow(!show)}>Toggle</button>
      {show && <p>Hidden Text</p>}
    </div>
  );
}
\end{lstlisting}
\end{enumerate}
\end{tcolorbox}

\subsubsection{Hooks (ES6+)}
\begin{itemize}
    \item \textbf{\texttt{useState}}: Adds state to functional components.
    \item \textbf{\texttt{useEffect}}: Handles side effects (like data fetching). Replaces lifecycle methods like \texttt{componentDidMount}.
\end{itemize}

\subsubsection{Data Fetching (Axios)}
Axios is a popular library to fetch data from APIs.
\begin{lstlisting}[language=JavaScript]
useEffect(() => {
  axios.get('https://api.example.com/posts')
    .then(res => setPosts(res.data));
}, []);
\end{lstlisting}

\begin{tcolorbox}[colback=purple!5,colframe=purple!75,title=Practical Example: Data Fetching Hook]
Fetching user profile on mount:
\begin{lstlisting}[language=JavaScript]
import { useState, useEffect } from 'react';
import axios from 'axios';

function UserProfile({ userId }) {
  const [user, setUser] = useState(null);

  useEffect(() => {
    // Side effect: fetch data
    axios.get(`/api/users/${userId}`)
      .then(response => setUser(response.data))
      .catch(error => console.error(error));
  }, [userId]); // Runs when userId changes

  if (!user) return <div>Loading...</div>;
  return <h1>{user.name}</h1>;
}
\end{lstlisting}
\end{tcolorbox}



