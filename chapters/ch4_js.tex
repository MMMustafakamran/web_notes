\section{Chapter 4: JavaScript - The Language of the Web}

JavaScript is a versatile, high-level language used for both client-side and server-side development. This chapter covers everything from basic syntax to advanced ES6 features and DOM manipulation.

\begin{tcolorbox}[colback=red!5,colframe=red!75,title=Exam Tips \& Hints]
\begin{itemize}
    \item \textbf{Variables}: Difference between \texttt{var}, \texttt{let}, and \texttt{const} is a classic theoretical question. Mention block-scope vs function-scope.
    \item \textbf{Hoisting}: Remember that \texttt{var} declarations are hoisted but not initialized. This can lead to \texttt{undefined} being logged if a variable is accessed before its declaration within the same scope.
    \item \textbf{Equality}: Understand the difference between \texttt{==} (loose, with coercion) and \texttt{===} (strict). Multiple MCQs often focus on this.
    \item \textbf{Array Methods}: \texttt{map()}, \texttt{filter()}, and \texttt{reduce()} are highly important. Be careful with \texttt{reverse()} and \texttt{sort()} as they modify the original array in-place.
    \item \textbf{Types}: Remember that \texttt{typeof []} is \texttt{"object"}.
\end{itemize}
\end{tcolorbox}

\subsection{JavaScript Fundamentals}
\subsubsection{Variables}
\begin{itemize}
    \item \textbf{\texttt{var}}: Function-scoped, can be redeclared and hoisted.
    \item \textbf{\texttt{let}}: Block-scoped, cannot be redeclared, not hoisted (preferred).
    \item \textbf{\texttt{const}}: Block-scoped, constant value (cannot be reassigned).
\end{itemize}

\subsubsection{Data Types}
\begin{itemize}
    \item \textbf{Primitive}: \texttt{String}, \texttt{Number}, \texttt{Boolean}, \texttt{Undefined}, \texttt{Null}, \texttt{Symbol}, \texttt{BigInt}.
    \item \textbf{Complex}: \texttt{Object}, \texttt{Array}, \texttt{Function}.
    \item \textit{Note:} \texttt{typeof null} returns \texttt{"object"} (a known bug).
\end{itemize}

\subsubsection{Operators}
\begin{itemize}
    \item \textbf{Assignment}: \texttt{=}, \texttt{+=}, \texttt{-=}, etc.
    \item \textbf{Comparison}: \texttt{==} (equal value), \texttt{===} (equal value and type), \texttt{!=}, \texttt{!==}.
    \item \textbf{Logical}: \texttt{\&\&} (AND), \texttt{||} (OR), \texttt{!} (NOT).
\end{itemize}

\subsubsection{Control Structures}
\begin{itemize}
    \item \textbf{Conditionals}: \texttt{if}, \texttt{else if}, \texttt{else}, \texttt{switch}.
    \item \textbf{Loops}: 
    \begin{itemize}
        \item \texttt{for}: Classic iterator.
        \item \texttt{for...in}: Iterates over object keys.
        \item \texttt{for...of}: Iterates over iterable values (like array elements).
        \item \texttt{while} \& \texttt{do...while}.
    \end{itemize}
\end{itemize}

\subsection{Strings and Arrays}
\subsubsection{String Methods}
Common methods: \texttt{concat()}, \texttt{charAt()}, \texttt{replace()}, \texttt{toLowerCase()}, \texttt{toUpperCase()}, \texttt{trim()}, \texttt{split()}.

    \item \textbf{Ordering}: \texttt{sort()} (alphabetical), \texttt{reverse()}.
\end{itemize}

\begin{tcolorbox}[colback=pink!5,colframe=pink!75,title=Practical Example: Functional Programming]
Using \texttt{map} and \texttt{filter} to process data:
\begin{lstlisting}[language=JavaScript]
const products = [
  { name: 'Laptop', price: 1200 },
  { name: 'Phone', price: 800 },
  { name: 'Tablet', price: 500 }
];

// 1. Filter products cheaper than 1000
const affordable = products.filter(p => p.price < 1000);

// 2. Map to an array of names
const names = affordable.map(p => p.name); 

console.log(names); // ['Phone', 'Tablet']
\end{lstlisting}
\end{tcolorbox}

\subsection{Functions}
\subsubsection{Definitions}
\begin{enumerate}
    \item \textbf{Declaration}: \texttt{function name() \{ ... \}} (Hoisted)
    \item \textbf{Expression}: \texttt{const name = function() \{ ... \}} (Anonymous or named)
    \item \textbf{Arrow Function}: \texttt{const name = () =$>$ \{ ... \}} (Shorter syntax, no \texttt{this} binding)
\end{enumerate}

\subsubsection{Parameters and Arguments}
\begin{itemize}
    \item \textbf{Default Parameters}: \texttt{function greet(name = "Guest") \{ ... \}}
    \item \textbf{Arguments Object}: Array-like object containing all arguments passed to a function.
    \item \textbf{Rest Parameters}: \texttt{function sum(...nums) \{ ... \}} (Collects remaining arguments into an array).
\end{itemize}

\subsection{Object-Oriented JavaScript (Classes)}
Classes are syntactical sugar over prototypes.
\begin{lstlisting}[language=JavaScript]
class Rectangle {
  constructor(height, width) {
    this.height = height;
    this.width = width;
  }
  // Getter
  get area() { return this.height * this.width; }
  // Setter
  set side(val) { this.height = val; this.width = val; }
}
const square = new Rectangle(10, 10);
\end{lstlisting}

\begin{tcolorbox}[colback=brown!5,colframe=brown!75,title=Practical Example: Class Inheritance]
Creating specialized classes from a base class:
\begin{lstlisting}[language=JavaScript]
class Animal {
  constructor(name) { this.name = name; }
  speak() { console.log(`${this.name} makes a noise.`); }
}

class Dog extends Animal {
  speak() { console.log(`${this.name} barks!`); }
}

const d = new Dog('Mitzie');
d.speak(); // Mitzie barks!
\end{lstlisting}
\end{tcolorbox}

\subsection{The Document Object Model (DOM)}
The DOM is a programming interface for HTML documents. It represents the page as a tree of objects.

\subsubsection{Selecting Elements}
\begin{itemize}
    \item \texttt{document.getElementById('id')}
    \item \texttt{document.querySelector('.selector')}
    \item \texttt{document.querySelectorAll('.selector')}
\end{itemize}

\subsubsection{Manipulating Elements}
\begin{itemize}
    \item \textbf{Content}: \texttt{element.innerHTML}, \texttt{element.textContent}, \texttt{element.value} (for inputs).
    \item \textbf{Attributes}: \texttt{getAttribute()}, \texttt{setAttribute()}, \texttt{removeAttribute()}, \texttt{hasAttribute()}.
    \item \textbf{Styles}: \texttt{element.style.color = 'red';} (uses camelCase for CSS properties).
    \item \textbf{Creation}: \texttt{document.createElement('div')}, \texttt{parent.appendChild(child)}.
\end{itemize}

\subsection*{Comprehensive Exam Prep: JavaScript}

\begin{tcolorbox}[colback=yellow!5,colframe=yellow!75,title=Past Paper Questions]
\begin{enumerate}
    \item \textbf{MCQ:} What will \texttt{console.log(typeof [])} output?
    \textbf{Answer:} object (Arrays are objects in JS).

    \item \textbf{Code Analysis (Sessional 2 2024):}
    \begin{lstlisting}[language=JavaScript]
var arr1 = "web".split('');
var arr2 = arr1.reverse();
var arr3 = "Programming".split('');
arr2.push(arr3);
console.log("array 2: " + arr2);
    \end{lstlisting}
    \textbf{Explanation:} 
    1. \texttt{arr1} = ['w','e','b']. 
    2. \texttt{arr2} reference refers to \texttt{arr1} reversed: ['b','e','w']. Note: \texttt{reverse()} mutates in place if referencing same array, but here \texttt{split} creates new.
    3. \texttt{arr3} = ['P','r',...]. 
    4. \texttt{push} adds the \textit{entire array} \texttt{arr3} as a single element at the end.
    \newline \textbf{Output:} \texttt{b,e,w,P,r,o,g,r,a,m,m,i,n,g} (Browsers flatten for display string).

    \item \textbf{DOM Manipulation:} Write code to display user name N times.
    \begin{lstlisting}[language=JavaScript]
function displayName() {
  const num = document.getElementById('num').value;
  const list = document.getElementById('list');
  list.innerHTML = ''; // Clear previous
  for(let i=0; i<num; i++) {
     const li = document.createElement('li');
     li.textContent = "My Name";
     list.appendChild(li);
  }
}
    \end{lstlisting}
\end{enumerate}
\end{tcolorbox}

\begin{tcolorbox}[colback=gray!5,colframe=gray!75,title=Extra Practice Questions]
\subsubsection*{ES6 Features}
\begin{enumerate}[resume]
    \item Which keyword allows declaring a variable that cannot be reassigned?
    \textbf{Answer:} \texttt{const}

    \item Rewrite this using Arrow Functions:
    \begin{lstlisting}[language=JavaScript]
    function sum(a, b) { return a + b; }
    \end{lstlisting}
    \textbf{Answer:} \texttt{const sum = (a, b) => a + b;}
\end{enumerate}

\subsubsection*{Advanced Scoping}
\textbf{Question:} Output of the following?
\begin{lstlisting}[language=JavaScript]
console.log(a);
var a = 5;
\end{lstlisting}
\textbf{Answer:} \texttt{undefined}. (Reason: \texttt{var} declarations are hoisted, but initializations are not).
\end{tcolorbox}

\begin{lstlisting}[language=JavaScript]
const btn = document.querySelector('#myBtn');
btn.addEventListener('click', (event) => {
  console.log('Button clicked!', event.target);
});
\end{lstlisting}

\begin{tcolorbox}[colback=orange!5,colframe=orange!75,title=Practical Example: Dynamic DOM Update]
Changing text and style on a button click:
\begin{lstlisting}[language=JavaScript]
const heading = document.querySelector('h1');
const colorBtn = document.querySelector('#colorBtn');

colorBtn.addEventListener('click', () => {
  heading.textContent = 'Color Changed!';
  heading.style.color = 'blue';
  
  // Create a new element
  const p = document.createElement('p');
  p.innerText = 'This was added dynamically.';
  document.body.appendChild(p);
});
\end{lstlisting}
\end{tcolorbox}

\subsection*{Comprehensive Exam Prep: JavaScript}

\begin{tcolorbox}[colback=yellow!5,colframe=yellow!75,title=Past Paper Questions]
\begin{enumerate}
    \item \textbf{MCQ:} What will \texttt{console.log(typeof [])} output?
    \textbf{Answer:} object (Arrays are objects in JS).

    \item \textbf{Code Analysis (Sessional 2 2024):}
    \begin{lstlisting}[language=JavaScript]
var arr1 = "web".split('');
var arr2 = arr1.reverse();
var arr3 = "Programming".split('');
arr2.push(arr3);
console.log("array 2: " + arr2);
    \end{lstlisting}
    \textbf{Explanation:} 
    1. \texttt{arr1} = ['w','e','b']. 
    2. \texttt{arr2} reference refers to \texttt{arr1} reversed: ['b','e','w']. Note: \texttt{reverse()} mutates in place if referencing same array, but here \texttt{split} creates new.
    3. \texttt{arr3} = ['P','r',...]. 
    4. \texttt{push} adds the \textit{entire array} \texttt{arr3} as a single element at the end.
    \newline \textbf{Output:} \texttt{b,e,w,P,r,o,g,r,a,m,m,i,n,g} (Browsers flatten for display string).

    \item \textbf{DOM Manipulation:} Write code to display user name N times.
    \begin{lstlisting}[language=JavaScript]
function displayName() {
  const num = document.getElementById('num').value;
  const list = document.getElementById('list');
  list.innerHTML = ''; // Clear previous
  for(let i=0; i<num; i++) {
     const li = document.createElement('li');
     li.textContent = "My Name";
     list.appendChild(li);
  }
}
    \end{lstlisting}
\end{enumerate}
\end{tcolorbox}

\begin{tcolorbox}[colback=gray!5,colframe=gray!75,title=Extra Practice Questions]
\subsubsection*{ES6 Features}
\begin{enumerate}[resume]
    \item Which keyword allows declaring a variable that cannot be reassigned?
    \textbf{Answer:} \texttt{const}

    \item Rewrite this using Arrow Functions:
    \begin{lstlisting}[language=JavaScript]
    function sum(a, b) { return a + b; }
    \end{lstlisting}
    \textbf{Answer:} \texttt{const sum = (a, b) => a + b;}
\end{enumerate}

\subsubsection*{Advanced Scoping}
\textbf{Question:} Output of the following?
\begin{lstlisting}[language=JavaScript]
console.log(a);
var a = 5;
\end{lstlisting}
\textbf{Answer:} \texttt{undefined}. (Reason: \texttt{var} declarations are hoisted, but initializations are not).
\end{tcolorbox}




