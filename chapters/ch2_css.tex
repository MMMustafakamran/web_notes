\section{Chapter 2: CSS - Cascading Style Sheets}

\begin{tcolorbox}[colback=red!5,colframe=red!75,title=Exam Tips \& Hints]
\begin{itemize}
    \item \textbf{Specificity}: High-probability topic. Remember: Inline Styles $>$ ID $>$ Class/Attribute $>$ Element.
    \item \textbf{Box Model}: Understand that \texttt{padding} is inside the border and \texttt{margin} is outside.
    \item \textbf{Selectors}: Be comfortable with pseudo-classes like \texttt{:hover} and \texttt{:nth-child()}.
\end{itemize}
\end{tcolorbox}

This document covers the fundamentals and advanced properties of CSS, used to control the visual presentation of web pages.

\subsection{What is CSS?}
CSS stands for \textbf{Cascading Style Sheets}.
\begin{itemize}
    \item \textbf{History}: Created by Hakon Lie in 1994; now a W3C standard.
    \item \textbf{Purpose}: Controls layout, enforces uniformity, saves time, and enables multiple device compatibility.
    \item \textbf{Rules}: A CSS rule consists of a \textbf{Selector} and a \textbf{Declaration Block}.
    \begin{itemize}
        \item \texttt{selector \{ property: value; \}}
    \end{itemize}
\end{itemize}

\subsection{Inserting CSS}
There are three ways to apply CSS to an HTML document:

\begin{enumerate}
    \item \textbf{Inline Styles}: Added directly to an element using the \texttt{style} attribute.
    \begin{itemize}
        \item Example: \texttt{<h1 style="color:red;">Title</h1>}
    \end{itemize}
    \item \textbf{Internal/Embedded Styles}: Defined inside a \texttt{<style>} tag within the \texttt{<head>} section.
    \item \textbf{External Style Sheets}: Defined in a separate \texttt{.css} file and linked in the \texttt{<head>}.
    \begin{itemize}
        \item Example: \texttt{<link rel="stylesheet" type="text/css" href="mystyle.css">}
    \end{itemize}
\end{enumerate}

\textbf{Cascading Order}: Inline styles have the highest priority, followed by internal and external style sheets.

    \item \textbf{Attribute Selector}: Styles elements based on their attributes.
\end{itemize}

\begin{tcolorbox}[colback=purple!5,colframe=purple!75,title=Practical Example: Specific Selectors]
Different ways to target elements:
\begin{lstlisting}[language=CSS]
/* Tag: all paragraphs */
p { font-family: Arial; }

/* ID: unique header */
#main-header { background-color: navy; color: white; }

/* Class: reusable button */
.btn-submit { border-radius: 5px; cursor: pointer; }

/* Attribute: target specific inputs */
input[type="text"] { border: 1px solid gray; }
\end{lstlisting}
\end{tcolorbox}

    \item \textbf{Margin}: Transparent area outside the border (space between elements).
\end{enumerate}

\begin{tcolorbox}[colback=yellow!5,colframe=yellow!75,title=Practical Example: Box Model Visualization]
How properties combine to create the total size:
\begin{lstlisting}[language=CSS]
.box {
  width: 200px;    /* Content width */
  padding: 20px;   /* 20px on all sides */
  border: 5px solid black;
  margin: 15px;    /* Space outside the border */
}

/* Total width = 200 + 20(left) + 20(right) + 5(left border) + 5(right border) = 250px */
\end{lstlisting}
\end{tcolorbox}

\subsection{Visual Properties}
\subsubsection{Colors}
Colors can be specified by:
\begin{itemize}
    \item \textbf{Name}: \texttt{Tomato}, \texttt{DodgerBlue}.
    \item \textbf{RGB}: \texttt{rgb(255, 99, 71)}.
    \item \textbf{HEX}: \texttt{\#ff6347}.
\end{itemize}

\subsubsection{Fonts and Text}
\begin{itemize}
    \item \textbf{Font}: \texttt{font-family}, \texttt{font-size}, \texttt{font-weight}.
    \item \textbf{Text}: \texttt{text-align}, \texttt{text-decoration}, \texttt{text-transform}, \texttt{color}.
\end{itemize}

\subsubsection{Backgrounds}
\begin{itemize}
    \item \texttt{background-color}, \texttt{background-image}, \texttt{background-repeat} (repeat-x, repeat-y, no-repeat).
    \item \texttt{background-attachment} (fixed, scroll).
    \item \texttt{background-position} (e.g., \texttt{right top}).
    \item \textbf{Shorthand}: \texttt{background: \#ffffff url("img.png") no-repeat right top;}
\end{itemize}

\subsection{Advanced Styling}
\subsubsection{Links}
Links can be styled based on their state:
\begin{itemize}
    \item \texttt{a:link}, \texttt{a:visited}, \texttt{a:hover}, \texttt{a:active}.
\end{itemize}

\subsubsection{Borders and Outlines}
\begin{itemize}
    \item \textbf{Borders}: \texttt{border-style} (solid, dashed, dotted), \texttt{border-width}, \texttt{border-color}.
    \item \textbf{Outline}: A line drawn around elements, outside the borders, to make the element "stand out".
\end{itemize}

\subsection*{Comprehensive Exam Prep: CSS}

\begin{tcolorbox}[colback=red!5,colframe=red!75,title=Past Paper Questions (Sessional \& Final)]
\begin{enumerate}
    \item \textbf{MCQ:} Which CSS property is used to change text color?
    \begin{enumerate}[label=\alph*.]
        \item font-style
        \item color
        \item text-color
        \item background
    \end{enumerate}
    \textbf{Answer:} b) color

    \item \textbf{MCQ:} In CSS, which of the following selectors has the highest specificity?
    \begin{enumerate}[label=\alph*.]
        \item .card
        \item div p
        \item \#header
        \item p
    \end{enumerate}
    \textbf{Answer:} c) \#header (IDs have higher specificity than classes or tags).

    \item \textbf{Snippet:} Write CSS to style \texttt{.btn-primary} with navy background, white text, 10px 20px padding, and no border.
\begin{lstlisting}[language=CSS]
.btn-primary {
    background-color: navy;
    color: white;
    padding: 10px 20px;
    border: none;
}
\end{lstlisting}
\end{enumerate}
\end{tcolorbox}

\begin{tcolorbox}[colback=gray!5,colframe=gray!75,title=Extra Practice Questions]
\subsubsection*{Multiple Choice Questions}
\begin{enumerate}[resume]
    \item How do you make a list that lists its items with squares?
    \begin{enumerate}[label=\alph*.]
        \item \texttt{list: square;}
        \item \texttt{list-type: square;}
        \item \texttt{list-style-type: square;}
        \item \texttt{list-style-image: square;}
    \end{enumerate}
    \textbf{Answer:} c) \texttt{list-style-type: square;}

    \item What is the default value of the \texttt{position} property?
    \begin{enumerate}[label=\alph*.]
        \item relative
        \item fixed
        \item absolute
        \item static
    \end{enumerate}
    \textbf{Answer:} d) static
\end{enumerate}

\subsubsection*{Advanced Styling Concepts}
\textbf{Question:} Explain the CSS Box Model with a diagrammatic description.
\newline \textbf{Answer:} The Box Model consists of four layers surrounding the HTML element:
\begin{itemize}
    \item \textbf{Content}: The actual text or image.
    \item \textbf{Padding}: Transparent area between content and border.
    \item \textbf{Border}: A line going around the padding and content.
    \item \textbf{Margin}: Transparent area outside the border, separating the element from others.
\end{itemize}

\textbf{Question:} Center a \texttt{div} horizontally and vertically using Flexbox.
\begin{lstlisting}[language=CSS]
.container {
    display: flex;
    justify-content: center; /* Horizontal */
    align-items: center;     /* Vertical */
    height: 100vh;
}
\end{lstlisting}
\end{tcolorbox}


\begin{lstlisting}[language=CSS]
div {
  transition: width 2s, height 2s, transform 2s;
}
div:hover {
  width: 300px;
  transform: rotate(180deg);
}
\end{lstlisting}

\begin{tcolorbox}[colback=red!5,colframe=red!75,title=Practical Example: Hover Card Effect]
Creating an interactive card that scales up:
\begin{lstlisting}[language=CSS]
.card {
  width: 250px;
  background: white;
  transition: transform 0.3s ease-in-out, box-shadow 0.3s;
}

.card:hover {
  transform: scale(1.05); /* Enlarge slightly */
  box-shadow: 0 10px 20px rgba(0,0,0,0.2);
}
\end{lstlisting}
\end{tcolorbox}



