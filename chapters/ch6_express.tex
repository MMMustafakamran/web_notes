\section{Chapter 6: Express.js - Web Application Framework}

Express is a minimal and flexible Node.js web application framework that provides a robust set of features for web and mobile applications. It is "unopinionated," meaning it gives developers freedom in how they structure their apps.

\begin{tcolorbox}[colback=red!5,colframe=red!75,title=Exam Tips \& Hints]
\begin{itemize}
    \item \textbf{Middleware}: The \texttt{next()} function is critical. If you don't call it, the request hangs.
    \item \textbf{Routing}: Route parameters (e.g., \texttt{:id}) vs query strings (\texttt{?id=1}).
    \item \textbf{Methods}: Use \texttt{GET} for retrieval and \texttt{POST} or \texttt{PUT} for data submission/update.
    \item \textbf{RegEx Routing}: Practice the special characters (\texttt{?}, \texttt{+}, \texttt{*}) in route paths.
\end{itemize}
\end{tcolorbox}

\subsection{Getting Started}
\subsubsection{Setup}
\begin{enumerate}
    \item Create a directory: \texttt{mkdir myapp}
    \item Initialize npm: \texttt{npm init}
    \item Install Express: \texttt{npm install express}
\end{enumerate}

\subsubsection{Hello World Example}
\begin{lstlisting}[language=JavaScript]
const express = require('express');
const app = express();
const port = 3000;

app.get('/', (req, res) => {
  res.send('Hello World!');
});

app.listen(port, () => {
  console.log(`Server running on port ${port}`);
});
\end{lstlisting}

\subsection{Modules and Asynchrony}
\begin{itemize}
    \item \textbf{Modules}: Use \texttt{module.exports} to share code across files and \texttt{require()} to import it.
    \item \textbf{Async Programming}: Node is single-threaded and non-blocking. Use asynchronous APIs (callbacks, Promises) to avoid blocking the event loop.
    \item \textbf{Error-First Callbacks}: A convention where the first argument of a callback is the error object.
\end{itemize}

\subsection{Middleware}
Middleware functions are the backbone of Express. They have access to the request (\texttt{req}), response (\texttt{res}), and the \texttt{next} function in the application’s request-response cycle.

    \item \textbf{Third-party}: Community modules like \texttt{morgan} (logging) or \texttt{cors}.
\end{enumerate}

\begin{tcolorbox}[colback=gray!5,colframe=gray!75,title=Practical Example: Custom Logger Middleware]
A function that logs the request method and URL:
\begin{lstlisting}[language=JavaScript]
const logger = (req, res, next) => {
  console.log(`${new Date().toISOString()} - ${req.method} ${req.url}`);
  next(); // Pass control to the next handler
};

app.use(logger); // Apply to all routes
\end{lstlisting}
\end{tcolorbox}

    \item \textbf{Chaining}: Use \texttt{app.route()} to define multiple methods for a single path.
\end{itemize}

\begin{tcolorbox}[colback=blue!5,colframe=blue!75,title=Practical Example: URL Parameters]
Extracting data from the request URL:
\begin{lstlisting}[language=JavaScript]
// GET /shop/electronics/iphone
app.get('/shop/:category/:item', (req, res) => {
  const { category, item } = req.params;
  res.send(`Searching for ${item} in ${category}`);
});
\end{lstlisting}
\end{tcolorbox}

\subsection{Templating with EJS}
EJS (Embedded JavaScript) is a template engine that lets you generate HTML with plain JavaScript.
\begin{itemize}
    \item \textbf{Setup}: \texttt{app.set('view engine', 'ejs');}
    \item \textbf{Syntax}:
    \begin{itemize}
        \item \texttt{<\% \%>}: Executes logic (loops, conditionals).
        \item \texttt{<\%= \%>}: Outputs a value to the page.
    \end{itemize}
    \item \textbf{Rendering}: \texttt{res.render('index', \{ name: 'John' \});}
\end{itemize}

\begin{tcolorbox}[colback=green!5,colframe=green!75,title=Practical Example: RESTful API Implementation]
Standard CRUD endpoints for a "Task" resource:
\begin{lstlisting}[language=JavaScript]
let tasks = [{ id: 1, title: 'Learn Express' }];

app.get('/tasks', (req, res) => res.json(tasks));

app.post('/tasks', (req, res) => {
  const newTask = { id: Date.now(), ...req.body };
  tasks.push(newTask);
  res.status(201).json(newTask);
});

app.delete('/tasks/:id', (req, res) => {
  tasks = tasks.filter(t => t.id != req.params.id);
  res.status(204).send();
});
\end{lstlisting}
\end{tcolorbox}

\subsection*{Comprehensive Exam Prep: Express.js}

\begin{tcolorbox}[colback=magenta!5,colframe=magenta!75,title=Past Paper Questions]
\begin{enumerate}
    \item \textbf{MCQ:} Which method is used to define a GET route in Express?
    \textbf{Answer:} \texttt{app.get()}

    \item \textbf{MCQ:} Which middleware function is used to parse incoming request bodies?
    \textbf{Answer:} \texttt{body-parser} (or built-in \texttt{express.json()} in modern versions).

    \item \textbf{Route Logic:} What is the output of the following sequence?
    \begin{lstlisting}[language=JavaScript]
    app.use((req, res, next) => { console.log('A'); next(); });
    app.get('/test', (req, res) => { console.log('B'); });
    \end{lstlisting}
    \textbf{Answer:} Output will be \texttt{A} then \texttt{B}. The middleware runs for all routes, then \texttt{next()} passes control to the specific GET handler.
\end{enumerate}
\end{tcolorbox}

\begin{tcolorbox}[colback=gray!5,colframe=gray!75,title=Extra Practice Questions]
\subsubsection*{Middleware \& Routing}
\textbf{Question:} Write a custom middleware that logs the Request URL and Method.
\begin{lstlisting}[language=JavaScript]
app.use((req, res, next) => {
    console.log(`Method: ${req.method}, URL: ${req.url}`);
    next(); // Critical to prevent hanging
});
\end{lstlisting}

\textbf{Question:} How do you access the query parameter \texttt{?id=123} in Express?
\newline \textbf{Answer:} \texttt{req.query.id}

\textbf{Question:} How do you access the URL parameter \texttt{/users/:uId} (where uId is 123)?
\newline \textbf{Answer:} \texttt{req.params.uId}
\end{tcolorbox}




